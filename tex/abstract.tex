%# -*- coding: utf-8-unix -*-
%%==================================================
%% abstract.tex for SJTU Master Thesis
%%==================================================

\begin{abstract}
NUMA(Non-Uniform Memory Access, 非一致性内存访问)架构的出现和普及克服了对称多处理器架构在扩展性方面的局限性,使得单台机器上能够容纳更多的计算核心,同时NUMA架构本身的设计也使得内存访问具有非一致性延迟的特征。NUMA架构在硬件层面的扩展使得共享内存的高并发应用能够生成更多的线程并将其分布到所有的核上来充分利用多核资源提高系统吞吐率;另一方面,内存及缓存的非一致性访问特性也使得挖掘和利用共享数据的局部性成为多线程获取更高性能的关键。对于极易成为系统性能瓶颈的锁来说,线程数的增多和内存访问的非一致性都使得充分挖掘和利用其局部性对于提高系统整体性能更为关键。

在NUMA架构的机器上,传统的可扩展性和性能都较好的MCS锁,CLH锁等基于队列的锁,由于内存和缓存访问的非一致性的特性,都出现了性能严重下降的问题,进而促进了局部性感知的层级锁,例如cohort locks,HMCS lock,的研究和发展。层级锁的本质是挖掘内存(包括缓存)访问的局部性,牺牲短期公平性来提高锁的性能,具体来说,通过改变锁的传递顺序使其尽可能地多在本地NUMA节点内传递来获取更高性能。由于层级锁高性能地获取要依赖于线程在NUMA节点(node)上的分布,因此线程的放置策略对于其最终能达到的性能提升有很大影响。现有层级锁中线程通常会被紧凑(compact)地放置在尽可能少的节点上来保持更多的局部性。这种线程放置策略有利于层级锁挖掘和利用局部性获取更高的性能,但是层级锁优先本地传递的传递机制使得其在牺牲短期公平性的同时也不能保证层级锁的长期公平性。除此以外,将线程平均(even)放置在相关节点上能够保证层级锁的长期公平性,但是由于线程分布较为分散所以在锁的竞争不是很激烈的时候相比紧凑放置会存在严重的性能下降问题。

现有地线程放置策略或者不能同时保证层级锁地性能和长期公平性,或者不能在特定竞争强度下同时保证性能和长期公平性,然而在很多应用场景比如公有云中性能和公平性都是缺一不可的。为了克服上述两种线程放置策略的不足,在这篇文章中我们提出了一种新的线程放置框架MSS。对于紧凑放置,我们引入了一种轻量级的线程轮换(shift)机制来以极小的性能损失为代价抹平线程之间的吞吐率差异从而使得紧凑策略能够保证层级锁的长期公平性,我们称新的放置策略为有轮换的紧凑放置(compact with shift);对于平均放置,为了尽可能地降低性能损失,我们限制其所能使用的节点数为能放下所有线程的最小节点数而不是所有节点,我们称新的放置策略为加强地平均放置(restricted even)。MSS本质上是基于上述两种改进策略的,能根据应用中层级锁的竞争状况动态调整线程放置策略的一种混合解决方案。它通过定期地取样每个线程的锁相关的事件来评估当前层级锁的竞争状况,并且借用了Malthusian locks的饱和点的概念来根据锁的竞争状况调节线程的放置策略。总的来说,MSS是一种竞争感知的混合式的线程放置框架,它在层级锁的竞争强度足够高的时候应用加强地平均放置策略,否则应用有轮换地紧凑放置策略,从而不论锁的竞争强度如何变化都能以最小的额外代价来同时保证性能和长期公平性。

\keywords{\large 层级锁 \quad 竞争感知的 \quad 长期公平性}
\end{abstract}

\begin{englishabstract}
The popularity of NUMA (Non-Uniform Memory Access) architecture has stimulated interest in designing locality-preserving hierarchical locks, such as HMCS and cohort locks. Essentially these hierarchical locks achieve higher throughput by exploiting locality and thus they tend to place threads across sockets in a compact way(compact placement) to preserve more locality. While this placement strategy could lead to severe long-term unfairness among threads due to the local-preferred lock transfer mechanism. Another threads placement strategy, even placement, which places threads evenly across all associated sockets, avoids the long-term unfairness problem but could lead to throughput degradation if contention is not high enough.

To find the reason behind the behaviors of above threads placement, we take a two-level MCS lock as an example and model the throughput and long-term fairness of the lock. We first model throughput of the lock from the aspects of lock transfer and find that the key to ensure high throughput is to place enough number of threads on each socket to saturate each lock MCS lock. We then model long-term fairness of the lock by classifying all threads into several equivalent groups according to their throughput theoretically in the long run, we analyze and conclude that two threads running on the same socket are in the same equivalent group, while for two threads running on different sockets, if and only if there are same number of threads running on the two sockets are the two threads in the same equivalent group. we find that one sufficient condition to guarantee long-term fairness of the lock is to make all threads in the same equivalent group, that is to place same number of threads on each socket. These two models explain why compact placement could not deliver long-term fairness and why even placement could not achieve high throughput in some cases.

Based on existing threads placement strategies and aboving modeling, in this paper, we present a new threads placement framework MSS. Basically, MSS is a hybrid solution of compact placement and even placement which could dynamically apply a most suitable threads placement strategy according to lock contention level. MSS profiles lock contention of the application by sampling lock events periodically and borrows the concept of saturation point from Malthusian locks to help decide and switch to if necessary a most suitable threads placement strategy for current lock contention level. Besides, MSS introduces a light-weight technique called "shift" to facilitate compact placement to avoid the long-term unfairness problem, and restricts even placement to use minimum required sockets to avoid throughput degradation. The result is a contention-aware hybrid threads placement strategy which applies restricted even placement under high contention and compact placement with shift otherwise, so that it could achieve both high throughput and long-term fairness regardless of the lock contention level with as little additional overhead as possible.

\englishkeywords{\large hierarchical locks, threads placement, contention-aware}
\end{englishabstract}

