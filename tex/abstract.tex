%# -*- coding: utf-8-unix -*-
%%==================================================
%% abstract.tex for SJTU Master Thesis
%%==================================================

\begin{abstract}
NUMA(Non-Uniform Memory Access, 非一致性内存访问)架构的出现和普及克服了对称多处理器架构在扩展性方面的局限性,同时也引入了内存访问的非一致性延迟。NUMA架构的这种新特性使得传统锁在其上的性能严重下降,进而促进了局部性感知的层级锁,例如cohort locks,HMCS lock,的研究和发展。层级锁的本质是挖掘内存(包括缓存)访问的局部性来提高锁的性能,所以使用层级锁时线程通常会被紧凑(compact)的放置在尽可能少的节点(node)上来保持更多的局部性。这种线程放置策略有利于层级锁获取更高的性能,但是不能保证层级锁的长期公平性。将线程平均(even)放置在相关节点上能够保证层级锁的长期公平性,但是这种放置策略在锁的竞争不是很激烈的时候存在严重的性能下降问题。

为了克服上述两种线程放置策略的不足,在这篇文章中我们提出了一种新的线程放置框架MSS。对于compact策略,我们引入了一种轻量级的“shift”机制以极小的性能损失为代价抹平线程之间的吞吐率差异从而使得compact策略能够保证层级锁的长期公平性(compact with shift);对于even策略,为了尽可能地降低性能损失,我们限制其所能使用地节点数为放下所有线程的最小节点数而不是所有节点(restricted even)。MSS本质上是基于上述两种改进策略的,能根据应用中层级锁的竞争状况动态调整线程放置策略的一种混合解决方案。它通过定期地取样每个线程的锁相关的事件来评估当前层级锁的竞争状况,并且借用了Malthusian locks的饱和点的概念来根据锁的竞争状况调节线程的放置策略。总的来说,MSS是一种竞争感知的混合式的线程放置框架,它在层级锁的竞争强度足够高的时候应用restrict even策略,否则应用compact with shift策略,从而不论锁的竞争强度如何变化都能以最小的额外代价来同时保证性能和长期公平性。

\keywords{\large 层级锁 \quad 竞争感知的 \quad 长期公平性}
\end{abstract}

\begin{englishabstract}

The popularity of NUMA (Non-Uniform Memory Access) architecture has stimulated interest in designing locality-preserving hierarchical locks, such as HMCS and cohort locks. Essentially these hierarchical locks achieve higher throughput by exploiting locality and tend to place threads across sockets in a compact way(compact placement) to preserve more locality. While this placement strategy could lead to severe long-term unfairness among threads due to the local-preferred lock transfer mechanism. Another threads placement strategy, even placement, which places threads evenly across all associated sockets, avoids the long-term unfairness problem but could lead to throughput degradation if contention is not high enough.

To overcome disadvantages of above threads placement strategies, in this paper, we present a new threads placement framework MSS. Basically, MSS is a hybrid solution of compact placement and even placement which could dynamically apply a most suitable threads placement strategy according to lock contention level. MSS profiles lock contention of the application by sampling lock events periodically and borrows the concept of saturation point from Malthusian locks to help decide and switch to if necessary a most suitable threads placement strategy for current lock contention level. Besides, MSS introduces a light-weight technique called "shift" to facilitate compact placement to avoid the long-term unfairness problem, and restricts even placement to use minimum required sockets to avoid throughput degradation. The result is a contention-aware hybrid threads placement strategy which applies restricted even placement under high contention and compact placement with shift otherwise, so that it could achieve both high throughput and long-term fairness regardless of the lock contention level with as little additional overhead as possible.

\englishkeywords{\large hierarchical locks, threads placement, contention-aware}
\end{englishabstract}

