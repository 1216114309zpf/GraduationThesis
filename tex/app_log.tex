%# -*- coding: utf-8-unix -*-
%% app2.tex for SJTU Master Thesis
%% based on CASthesis
%% modified by wei.jianwen@gmail.com
%% version: 0.3a
%% Encoding: UTF-8
%% last update: Dec 5th, 2010
%%==================================================

\chapter{锁替换技术}
\section{锁替换技术}
由于Pthread(POSIX Threads)线程库具有很好的平台可移植性,所以很多历史遗留的多线程应用,比如Memcached,都采用了Pthread线程模型,相应地Pthread mutex也就成了这些应用中最常用的锁。为了在这些应用中测试和使用其他锁并且省去冗长且容易出错的手动替换的麻烦,Hugo Guiroux\cite{guiroux2016multicore}开发了LiTL(Library for Transparent Lock interposition)。

LiTL是一个Linux/x86平台下可以在运行时将基于Pthread mutex的应用中的Pthread mutex替换为其他锁的库。为了实现锁替换,LiTL使用一个可扩展的并发哈希表(CLHT\cite{david2015asynchronized})维护了一个标准Pthread锁(pthread\_mutex\_t)实例和其他用来替换Pthread mutex的锁实例(比如MCS锁)的映射,也就是说LiTL必须追踪Pthread mutex从pthread\_mutex\_init()到pthread\_mutex\_destroy()的整个生命周期,并且在该生命周期中,每次pthread\_mutex\_lock()都会必须触发一个在上述映射中查找对应其他锁实例的查找操作。此外,某些锁的lock/unlock接口除了锁变量本身以外还需要其他参数,比如在MCS锁中该额外参数对应每个线程的记录(record),对于这些锁,LiTL还在映射中为每个锁实例每个线程维护了一个额外的结构体来表示这些额外参数。

LiTL使用LD\_PRELOAD来拦截基于Pthread mutex的应用中类似pthread\_mutex\_*这样的函数调用,然后查询CLHT,找到对应的映射锁实例,调用对应的函数。LD\_PRELOAD利用了Linux系统中动态链接器(dynamic linker)提供的功能来使用户可以指定动态链接器在加强其他共享库之前绑定某个库的符号(symbol),LiTL提供了与Pthread库相同的外部接口,所以替换Pthread mutex的锁对应的库通过LD\_PRELOAD在应用运行时优先加载后,所有Pthread mutex相关的调用最终都会被拦截到LiTL提供的库中。