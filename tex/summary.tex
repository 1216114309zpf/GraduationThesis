%# -*- coding: utf-8-unix -*-
%%==================================================
%% conclusion.tex for SJTUThesis
%% Encoding: UTF-8
%%==================================================

\begin{summary}
本文研究了线程放置策略对基于队列的层级锁的性能和长期公平性的影响,并且从线程放置的角度对现有的层级锁实现进行了改进,使其能够在复杂多变的锁竞争状况下同时保证高性能和长期公平性。本文的研究的直接基础是基于队列的层级锁,而基于队列的层级锁的出现是为了解决NUMA架构下基于队列的锁在高并发并行共享内存应用中性能衰退的问题。基于队列的层级锁在考虑底层NUMA因素的情况下通过在线程之间做锁的调度来改善基于队列的锁在NUMA架构下的性能,它的本质是通过牺牲短期公平性来提高性能,然而我们在实验中发现基于队列的层级锁在牺牲短期公平性的同时长期公平性也不能保证。

通过进一步地建模分析,我们认识到基于队列地层级锁的性能和长期公平性与线程在NUMA节点上的分布密切相关,其中性能主要与每个节点上的线程数是否超过层级锁当前的饱和点有关,而长期公平性主要与线程在节点间的分布是否均匀有关。基于此我们提出了竞争感知的混合线程放置框架CAH,CAH对层级锁中原有的两种线程放置策略分别做了改进,并且在此基础上通过感知应用中锁的竞争状况动态地应用最适合地线程放置策略,从能以尽可能小地代价保证了基于队列地层级锁能够在复杂多变地锁竞争变化情况下同时保证性能和长期公平性。

我们在libslock中的C-MCS锁中实现了CAH,并且按照litl库的统一接口对其进行了重写,添加了condition variable和trylock等功能,从而能够方便的利用加了CAH线程放置框架的C-MCS锁替代现有应用中的Pthread mutex。最后,我们分别在stress\_one和memcached两个benchmark上通过实验验证了本文提出的两种改进的线程放置策略及框架CAH在改进基于队列的层级锁的性能和长期公平性方面的有效性。

在后续工作中,我们打算对本文的研究做以下三个方面的改进:1)现有实现中主要考虑的是框架的功能方面,对于其性能方面的考虑不足,还可以从cache line对齐,减少变量的共享范围等方面对其做进一步的改进;2)现有实现只考虑文中实验平台的特性,移植到其他机器上需要大量的修改,所以我们打算将其中一些变量参数化从而改进其可移植性;3)现有实现只考虑了单个应用的情况,通常在云计算等环境下为了提高机器大的利用率一台机器上会运行多个类似应用,这种情况下需要考虑负载均衡等多种因素,将会更具挑战性。所以我们后续会考虑多个应用同时运行的情况对本文的研究做进一步改进。

\end{summary}
